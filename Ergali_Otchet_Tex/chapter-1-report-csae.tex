\chapter{\label{ch:ch01}ГЛАВА 1. ТЕОРЕТИЧЕСКАЯ ЧАСТЬ} % Нужно сделать главу в содержании заглавными буквами

\section{\label{sec:ch01/sec01}Техническое задание.}
\begin{enumerate}
	\item Кроссплатформенное приложение, способное запускаться на операционных системах Windows и GNU/Linux.
	\item Написание клеточного автомата "Игра жизнь" на платформе Unity, с использованием языка C\#.
	\item Возможность выбора карты и создания новых карт для клеточного автомата.
	\item Возможность генерации случайной карты.
        \item Возможность выбрать карту с неумирающей системой.
	\item Наличие в игре кнопки "Помощь", при нажатии на которую выводится информация о взаимодействии с игрой.
\end{enumerate}

\section{\label{sec:ch01/sec02}Игровой движок.}
Для создания кроссплатформенной программы на языке программирования C\# был использован игровой движок Unity

\textbf{Unity} --- это мощная кросс-платформенная среда разработки для создания широкого спектра интерактивного контента, включая видеоигры, обучающие программы, архитектурные визуализации и виртуальную реальность. Она предоставляет интегрированные инструменты для работы как с 2D, так и с 3D-графикой, что делает Unity популярным выбором среди разработчиков игр и интерактивных приложений различного масштаба — от независимых проектов до крупных игровых студий.

Основные особенности и преимущества Unity:
\begin{itemize}
    \item \textbf{Кросс-платформенность:} Unity поддерживает более 25 платформ, включая Windows, macOS, Linux, Android, iOS, WebGL, PlayStation, Xbox и многие другие, что позволяет разработчикам с легкостью портировать свои проекты на различные устройства и экосистемы;
    \item \textbf{Интуитивный интерфейс:} Unity обладает удобным и понятным пользовательским интерфейсом, который упрощает процесс разработки и позволяет даже начинающим разработчикам быстро освоиться в программе.;
    \item \textbf{Мощные инструменты для работы с графикой и анимацией:} В Unity встроены продвинутые инструменты для создания и редактирования 2D и 3D графики, анимаций, а также системы частиц, что делает возможным создание визуально привлекательных и технически сложных проектов;
    \item \textbf{Скриптование на C#:} Для создания логики игр и приложений в Unity используется язык программирования C#. Это обеспечивает гибкость и мощь в реализации функционала, делая возможным создание сложных систем взаимодействия и поведения объектов в игровом мире;
    \item \textbf{Активное сообщество и обширная база знаний:} Unity имеет одно из самых больших и активных сообществ разработчиков в мире. Это обеспечивает доступ к огромному количеству учебных материалов, руководств, а также готовых активов и плагинов, которые можно использовать в своих проектах;
    \item \textbf{Unity Asset Store:} Магазин активов Unity предлагает тысячи готовых ресурсов и инструментов, включая модели, текстуры, скрипты, инструменты для интеграции с другими сервисами и многое другое, что значительно ускоряет процесс разработки и позволяет сосредоточиться на уникальных аспектах проекта.
\end{itemize}


\section{\label{sec:ch01/sec03}Инструментарий.}
Для написания отчёта с помощью системы компьютерной верстки в \TeX была использована сайт Overleaf.

Для написания кода программы была использована IDE Microsoft Visual Studio.

Для работы с изображениями, используемых в ходе разработки программы, был использован графический редактор Adobe Photoshop 2023.

Для хранения проекта была выбрана система контроля версия GitHub~\cite{wikiRUGitHub}.

Проверка работоспособности и сборка программы выполнялась на системе:
	\begin{itemize}
		\item \textbf{OC}: \textit{Windows 10}
		\item \textbf{ЦП}: \textit{AMD Ryzen 5 4600H}
		\item \textbf{ОЗУ}: \textit{8gb}
		\item \textbf{Видеокарта}: \textit{NVIDIA GeForce GTX 1650 Ti}
	\end{itemize}

\section{\label{sec:ch01/sec04}Клеточный автомат <<Игра жизнь>>.}
\subsection{\label{subsec:ch01/sec04/subsec01}История клеточного автомата <<Игра жизнь>>.}

<<Игра жизнь>> — это клеточный автомат, созданный британским математиком Джоном Конвеем в 1970 году. Эта <<игра>> стала одним из самых известных примеров клеточного автомата и занимает особое место в теории вычислительных систем и математической биологии.

История создания <<Игры жизнь>> началась с желания Конвея изучить возможности простых математических моделей для имитации жизни и эволюции. Интересно, что, несмотря на простоту правил, <<Игра жизнь>> способна продемонстрировать чрезвычайно сложное и непредсказуемое поведение, что делает ее удивительным примером эмерджентности — появления сложных структур и паттернов из простых взаимодействий. Вместе с коллегами из Кембриджского университета он разработал первые версии игры, которые изначально испытывали на досках для шахмат, заполняя клетки <<живыми>> или <<мертвыми>> состояниями и просчитывая следующие поколения вручную.

Опубликованная в октябре 1970 года в колонке Мартина Гарднера в журнале <<Scientific American>>, <<Игра жизнь>> моментально привлекла внимание широкой аудитории. Люди были поражены идеей, что такие простые правила могут породить бесконечное множество удивительных форм и поведенческих паттернов. В то время как некоторые находили в ней отражение эволюционных процессов и даже возможность изучения искусственного интеллекта, другие видели в <<Игре жизнь>> искусство и новую форму развлечения.

Важной вехой в истории <<Игры жизнь>> стало ее распространение среди первых пользователей персональных компьютеров. Программные реализации игры позволили автоматизировать вычисления и наблюдать за развитием системы в реальном времени, что существенно углубило исследования и способствовало обнаружению всё новых и новых удивительных структур, таких как <<планеры>> (космические корабли), <<планерные пушки>> и самовоспроизводящиеся конструкции.

<<Игра жизнь>> сыграла значительную роль в развитии теории клеточных автоматов и комплексных систем, вдохновив множество ученых и разработчиков на создание собственных моделей и исследования в области искусственной жизни, системной биологии и даже криптографии. Она продолжает оставаться популярной не только среди ученых, но и среди художников, программистов и любителей загадок, представляя собой универсальный язык для исследования сложности из простоты.

\subsection{\label{subsec:ch01/sec04/subsec02}Правила игры.}
Правила <<Игры жизнь>> Джона Конвея просты, но в то же время обладают удивительной глубиной, позволяя моделировать сложные динамические процессы. <<Игра>> происходит на бесконечной двумерной сетке клеток, каждая из которых может находиться в одном из двух состояний: быть <<живой>> или <<мертвой>>. Состояние каждой клетки в следующем поколении определяется ее текущим состоянием и состояниями восьми ее соседей по горизонтали, вертикали и диагоналям.

Правила перехода из поколения в поколение следующие:
	\begin{itemize}
		\item \textbf{Рождение}: Мертвая клетка становится живой в следующем поколении, если ровно три из ее восьми соседей живы в текущем поколении. Это правило символизирует репродуктивное "рождение" из стабильной, но не перенаселенной среды;
		\item \textbf{Смерть от одиночества}: Живая клетка становится мертвой в следующем поколении, если два или менее из ее восьми соседей живы в текущем поколении. Это моделирует смерть из-за недостатка "социальных" взаимодействий;
		\item \textbf{Выживание}: Живая клетка остается живой в следующем поколении, если у нее два или три живых соседа. Это условие обеспечивает оптимальную среду для устойчивого существования;
		\item \textbf{Смерть от перенаселения}: Живая клетка становится мертвой в следующем поколении, если четыре и более из ее восьми соседей живы в текущем поколении. Это представляет смерть из-за слишком высокой конкуренции за ресурсы.
	\end{itemize}

Несмотря на кажущуюся простоту, эти правила порождают удивительно разнообразные и часто непредсказуемые паттерны, включая стационарные фигуры, осциллирующие структуры, которые повторяются через определенное количество поколений, и даже <<космические корабли>>, перемещающиеся по сетке. Эти паттерны и их взаимодействия исследуются в рамках <<Игры жизнь>>, демонстрируя сложность, возникающую из простых начал.

